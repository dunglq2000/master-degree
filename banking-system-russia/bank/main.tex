\documentclass{article}

\usepackage{tabto}
\newcommand{\mytab}{\tabto{1cm}}

\usepackage{tabularx}

\usepackage{hyphenat}
\usepackage{enumitem}
\usepackage{indentfirst}
\usepackage{geometry}
\usepackage{amssymb, amsmath, amsthm}
\usepackage{titlesec}
\usepackage[fontsize=14pt]{fontsize}

% Пакет для содержания
\usepackage{tocloft}
\usepackage{hyperref}

% Language
\usepackage{fontspec}
\usepackage{polyglossia}
\usepackage{csquotes}
\setmainlanguage{russian}
\newfontfamily{\cyrillicfonttt}{Times New Roman}
\setmainfont{Times New Roman}

\geometry{a4paper, top=2cm, left=2cm, bottom=2cm, right=1cm, includefoot, foot=14pt}

\renewcommand{\baselinestretch}{1.5}
%\setlength{\baselineskip}{1.5cm}
%\setstretch{1.5cm}
%\linespread{1.5}

\setlength{\parindent}{1.25cm}
% \setlist[itemize]{nosep,leftmargin=1.27cm,itemindent=0mm}
% \setlist[enumerate]{nosep,leftmargin=1.27cm,itemindent=0mm}

\makeatletter
\AddEnumerateCounter{\asbuk}{\russian@alph}{щ}
% \setlist[itemize]{nosep, leftmargin=0pt, labelindent=1.27cm, itemindent=*, label={-}}
\setlist[itemize]{nosep, leftmargin=0pt, labelindent=1.25cm, itemindent=*}
\setlist[enumerate]{nosep, leftmargin=0pt, labelindent=1.25cm, itemindent=*}

\titleformat{\section}[block]{\normalfont\normalsize\normalfont}{\hspace*{1.25cm}Глава~\thesection. }{0pt}{}{}
\titleformat{name=\section,numberless}[block]{\normalfont\normalsize\normalfont}{\hspace*{1.25cm}}{0pt}{}{}
\titleformat{\subsection}[block]{\normalsize\normalfont}{\hspace*{1.25cm}\thesubsection. }{0pt}{}{}

\titlespacing{\section}{0pt}{0pt}{0pt}
\titlespacing{\subsection}{0pt}{0pt}{0pt}

\hyphenpenalty=10000 \spaceskip=0.3em plus 2em minus 0.2em

\renewcommand{\contentsname}{Содержание}
\renewcommand{\figurename}{Рисунок}

%\renewcommand{\cfttoctitlefont}{\hspace{0.31\textwidth} \bfseries\Large}
\renewcommand{\cfttoctitlefont}{\normalsize}
\renewcommand{\cftsecfont}{\normalsize}

\renewcommand{\cftbeforetoctitleskip}{0em}
\renewcommand{\cftaftertoctitleskip}{0em}
\renewcommand{\cftsecpresnum}{Глава~}

\newlength\mylength
%\settowidth\mylength{\cftsecpresnum}
%\addtolength\cftsecnumwidth{\mylength}


% Строки с точками
\renewcommand{\cftsecleader}{\cftdotfill{\cftdotsep}}
% Точки после цифр в в содержании
\renewcommand{\cftsecaftersnum}{.}
\renewcommand{\cftsubsecaftersnum}{.}

% Подровнять subsection под точку главы
% (если глав будет больше десяти, будет чуть хуже)
\setlength{\cftsecindent}{0pt}
\setlength{\cftsubsecindent}{1.32cm}
% Интервал глав
\setlength{\cftbeforesecskip}{0pt}
%\renewcommand{\cftsecpagefont}{\normalfont}

\setlength{\cftsecnumwidth}{2.3cm}
\setlength{\cftsubsecnumwidth}{1cm}

\renewcommand{\cftsecpagefont}{}
\usepackage[parentracker=true,
  backend=biber,
  hyperref=auto,
  language=russian,
  autolang=other,
  citestyle=gost-numeric,
  defernumbers=true,
  bibstyle=gost-footnote,
  sorting=none
]{biblatex}

\usepackage{csquotes}

\addbibresource{references.bib}

\DeclareFieldFormat{labelnumberwidth}{#1\addperiod}
      
\defbibenvironment{gostbibliography}
  {\enumerate[noitemsep, labelindent=1.27cm, leftmargin=0cm, itemindent=*]
     %{\printtext[labelnumberwidth]{%
	%  \printfield{labelprefix}%
	  %\printfield{labelnumber}}}
     {}
     {\toggletrue{bbx:gostbibliography}%
      \renewcommand*{\revsdnamepunct}{\addcomma}%
      \renewcommand*{\labelnamepunct}{\addperiod\space}%
      \renewcommand*{\bibname}{\hspace{1.27cm}Список используемых источников}
      %\setlength{\leftmargin}{\bibhang}%
      %\setlength{\itemindent}{\leftmargin}%
      %\setlength{\itemsep}{\bibitemsep}%
      %\setlength{\parsep}{\bibparsep}
      %\setlength{\leftmargin}{0cm}
      %\setlength{\labelindent}{0cm}
      %\setlength{\itemsep}{\bibitemsep}
      }
      }
    {\endenumerate}
    {\item}

\DeclareFieldFormat{urldate}{(дата обращения #1)}

\usepackage{caption}
\usepackage{ragged2e}

\DeclareCaptionLabelSeparator{bar}{.}
\DeclareCaptionLabelSeparator{empt}{}
%\captionsetup[figure]{labelsep=bar}
%\captionsetup[table]{labelsep=endl,justification=raggedleft,singlelinecheck=off,skip=0pt}



\DeclareCaptionFormat{forfigure}
{%
    \hfill \\
    #1#2 #3
    %\hfill\\
}

\DeclareCaptionFormat{fortable}
{
    \raggedleft #1#2 \\ \justifying \indent #3
}

\captionsetup[figure]
{
  format=forfigure,
  labelsep=bar,
  justification=centering
  %skip=0pt
}

\captionsetup[table]
{
  format=fortable,
  labelsep=empt,
  %justification=raggedleft,
  singlelinecheck=off
}
%\setlength{\belowcaptionskip}{0\baselineskip}
\setlength{\belowcaptionskip}{0\parskip}

\usepackage{algpseudocode}
\usepackage{algorithm}

\usepackage{listings}

\lstset{basicstyle=\rmfamily,
  showstringspaces=false,
  commentstyle=\color{red},
  %keywordstyle=\color{blue}
  keywordstyle=\rmfamily,
  aboveskip=0pt,
  belowskip=0pt,
  %numbers=left,
}

\expandafter\def\expandafter\normalsize\expandafter{%
  %\setlength{\abovedisplayskip}{0pt}
  %\setlength{\belowdisplayskip}{0pt}
  %\setlength{\abovedisplayshortskip}{0pt}
  %\setlength{\belowdisplayshortskip}{0pt}
}

\usepackage{mdwlist}

\newcommand{\sectionA}[1]{
  \titleformat{\section}[block]{\normalfont\normalsize\raggedleft}{Приложение \thesection\\}{0pt}{}{}
  \section{#1}
}

\begin{document}

\begin{titlepage}
\begin{center}
Национальный исследовательский ядерный университет «МИФИ»

Кафедра 44 «Информационная безопасность банковских систем»
\end{center}

\vspace*{4\baselineskip}

\begin{center}
  Домашнее задание

  по дисциплине «Банковская система Российской Федерации»
\end{center}

\vspace*{4\baselineskip}

\noindent Тема: Банковский надзор и система ПОД/ФТ во Вьетнаме


\vspace*{4\baselineskip}

\begin{tabbing}
  \hspace{120pt}\=\hspace{240pt}\=\hspace{120pt}\=\kill
  Выполнил \> студент группы М24-705 \> Ле К.З. \\ 
  Преподаватель \> зав. кафедры 44 \> Толстой А.И.  \\
\end{tabbing}

\vspace*{4\baselineskip}
\begin{center}
  Москва, 2024
\end{center}
\end{titlepage}

\setcounter{page}{2}

\newpage

\tableofcontents

\newpage

\section*{Введение}
\addcontentsline{toc}{section}{Введение}

Отмывание денег и финансирование терроризма являются особенно серьезными проблемами, затрагивающими безопасность каждой страны. Каждая страна имеет свою собственную систему борьбы с отмыванием денег и финансированием терроризма. Каждая система имеет свои особенности в зависимости от национальной ситуации. Поэтому эта тема является актуальной.

Целями работы являются:

\begin{enumerate}
    \item Исследование банковского надзора и системы ПОД/ФТ во Вьетнаме и сравнение со системой в Российской Федерации.
    \item Получение практических навыков поиска и анализа информации, постановки и выполнения задач, формирования обоснованных выводов, подготовки отчета и публичной защиты результатов выполнения задания.
\end{enumerate}

Для достижения постановленных целей необходимо решить следующие задачи:

\begin{enumerate}
    \item Подбор и анализ информационных источников.
    \item Подготовка перечня терминов и определений.
    \item Анализ особенностей банковского надзора и системы ПОД/ФТ во Вьетнаме и в Российской Федерации.
    \item Проведение сравнения двух систем ПОД/ФТ.
\end{enumerate}

Работа состоит из введения, трёх глав, заключения и списка используемых источников, насчитывающего 17 наименований.

В первой главе проведен подбор и анализ нормативных и правовых документов в области предотвращения омывания денег и финансирования терроризма во Вьетнаме и в Российской Федерации.

Во второй главе рассматриваются термины и определения в области ПОД/ФТ.

Третья глава посвящена анализ сходства и различия между двум системами ПОД/ФТ.

\newpage

\section{Нормативная и правовая база выполнения домашнего задания}

\hfill

\subsection{Нормативная и правовая база во Вьетнаме}

Конституция Вьетнама является документом на верхнем уровне, регулирующим права и обязанности граждан. Поэтому, конституция играет важную роль в построении национальной системы ПОД/ФТ.

В документ \cite{law46} определено положение и функции Государственного банка Вьетнама (далее -- Государственный банк), его задачи и полномочия, организационная структура, руководство и штат, а также деятельность Государственного банка, и главное, банковской инспекции и надзора.

В следующей части документа рассматриваются цели, принципы, предметы, содержание банковской инспекции и надзора, а также права и обязанности объектов банковского надзора и ведение деятельности объектов банковской инспекции и надзора. В конце документа говорится о внутреннем аудите Государственного банка и вступлении в силу закона.

Документ \cite{law47} регулирует деятельность различных типов кредитных организаций, действующих во Вьетнаме. Следует отметить следующие 2 статьи:

Статья 207. Полномочия по проверке, инспекции и надзору

\begin{enumerate}
  \item Государственный банк имеет полномочия проверять, инспектировать и контролировать кредитные учреждения, филиалы иностранных банков, иностранные представительства в соответствии с положениями закона \textquote{О Государственном банке Вьетнама} и другими положениями соответствующих законов.
  \item Государственная инспекция осуществляет инспекцию кредитных организаций и филиалов иностранных банков в соответствии с законами об инспекции.
  \item На Министерство финансов возлагаются следующие обязанности:
  \begin{enumerate}[label=\asbuk*), labelindent=1.47cm]
    \item проверять, инспектировать и контролировать деятельность кредитных учреждений, филиалов иностранных банков, дочерних и дочерних компаний кредитных учреждений на рынке ценных бумаг и фондовом рынке в соответствии с положениями закона \textquote{О ценных бумагах} и других соответствующих правовых норм;
    \item проверять и контролировать страховую агентскую деятельность кредитных учреждений, филиалов иностранных банков, дочерних и дочерних компаний кредитных учреждений в соответствии с положениями закона \textquote{О страховом деле} и другими соответствующими законами;
    \item председательствовать, координировать и обмениваться информацией с Государственным банком при реализации положений пунктов а) и б) настоящего пункта.
  \end{enumerate}
  \item Министерства и ведомства министерского уровня в пределах своих функций, задач и полномочий осуществляют вневедомственные проверки, проверки и надзор за кредитными организациями, филиалами иностранных банков и иностранными представительствами.
\end{enumerate}

Статья 208. Права и обязанности объектов инспекции и надзора

\begin{enumerate}
  \item Предоставлять своевременную, полную и достоверную информацию и документы по запросу Государственного банка и других компетентных органов государственного управления в процессе проверки, инспекции и надзора.
  \item Нести ответственность за точность и правдивость предоставленной информации и документов.
  \item Обеспечить возможность подключения и доступа к онлайн-данным для обслуживания надзорной деятельности Государственного банка в соответствии с постановлениями управляющего Государственного банка.
  \item Сообщать и разъяснять рекомендации, предупреждения о рисках и операционной безопасности Государственного банка.
  \item Выполнять рекомендации, предупреждения о рисках и операционной безопасности Государственного банка.
  \item Выполнять заключения инспекции и исполнительные решения Государственного банка, Государственной инспекции и других органов в соответствии с положениями закона.
  \item Другие права и обязанности согласно положениям закона.
\end{enumerate}

Документ \cite{law14} является важным в предотвращении отмывания денег. Документ состоит из 4 глав:

\begin{itemize}[label={--}]
  \item первая глава определяет общие правила, субъекты отчетности, принципы ПОД, а также международное сотрудничества, национальную оценку рисков и запрещенные действия в ПОД;
  \item вторая глава рассматривает меры по ПОД. Она регулирует сбор, обновление и проверку информации о клиентах для классификации клиентов по уровню риска отмывания денег. Далее следует ответственность за создание внутренних правил и отчетности, предоставление и хранение информации и записей о ПОД. Далее идет раздел по сбору, обработке, анализу, обмену, предоставлению и передаче информации по ПОД. Наконец, документ рассматривает применение временных мер и устранении нарушений;
  \item треья глава определяет обязанности государственных органов по ПОД, в том числе: Правительство, Премьер-министр, Государственный банк, Министерство общественной безопасности, Министерство национальной обороны, Министерство финансов, Министерство строительства, Министерство юстиции, Министерство промышленности и торговли, Министерства планирования и инвестиций, Министерства внутренних дел, Министерства иностранных дел, Министерства информации и коммуникаций, других министерств и отраслей, а также обязанности народной прокуратуры и судов, и народных комитетов;
  \item четвертая глава содержит положения по реализации закона.
\end{itemize}

Документ \cite{tt0822} регламентирует порядок и процедуры банковского надзора. Банковский надзор включает два вида: микро- и макро-. В каждом виде надзора предусмотрены регламенты и инструкции по сбору и обработке документов, информации, данных, а также по контролю их содержания, составлению отчетов, отслеживанию записей и предложений, проведению корректирующих мероприятий. В следующей части документа определены контрольные меры в сфере банковского надзора и ответственность организаций и частных лиц, осуществляющих деятельность по банковскому надзору. В конце документа приведены образцы отчетов по банковскому надзору (микро и макро), а также некоторые соответствующие образцы.

Документ \cite{ndcp2614} определяет организацию, задачи и полномочия банковской инспекции и надзора. Далее в документе упоминается деятельность банковской инспекции и надзора, а также обязанности учреждений, организаций и частных лиц в ней.

Документ \cite{law28} определяет принципы предотвращения терроризма и борьбы с ним, национальную политику и ответственность в предотвращении терроризма. В целях организации антитеррористической деятельности создан национальный руководящий комитет, который выполнял задачи, предусмотренные настоящим законом. Далее документ рассматривает вопросы предотвращения терроризма и борьбы с ним, а также борьбы с финансированием терроризма и международного сотрудничества в этом. Наконец, закон определяет обязанности государственных органов, аналогичные документу \cite{law14}, в предотвращении терроризма.

\hfill

\subsection{Нормативная и правовая база в Российской Федерации}

Конституция Российской Федерации является документом на верхнем уровне, регулирующим права и обязанности граждан. Таким образом, конституция играет важную роль в построении системы ПОД/ФТ.

В \cite{fz86} определены положение, задачи и полномочия Банка России в области банковского надзора, а также организационная структура в сфере банковского надзора.

В \cite{pre808} показано, что ответственным органом в сфере ПОД/ФТ/ФРОМУ является Росфинмониторинг, а также указаны его функции и полномочия.

В соответствии с \cite{fz115} определены права и обязанности граждан, банка России, а также других ответственных органов в сфере предотвращения отмыванием денег и финансированием терроризма. Федеральный закон также предусматривает меры предотвращения отмываением денег. Кроме того, международное сотрудничество рассматрено для международной координации в вопросах борьбы с отмыванием денег и финансированием терроризма.

% Федеральный закон от 10.12.2003 N 173-ФЗ "О валютном регулировании и валютном контроле".

\hfill

\subsection{Международная нормативная и правовая база}

Конвенция ООН 1998 г. \cite{unc88} была ратифицирована как Вьетнамом, так и Российской Федерацией. Эта Конвенция особенно важна в международном сотрудничестве в борьбе с отмыванием денег и финансированием терроризма, поскольку большая часть доходов, полученных преступным путем, происходит от незаконного оборота запрещенных веществ.

% Конвенция СЕ об отмывании, выявлении, изъятии и конфискации доходов от преступной деятельности 1990 г. (статья 6)

Конвенция ООН 2000 г. \cite{unc00} также была ратифицирована Вьетнамом и Российской Федерацией. Предотвращение отмывания денег и финансирования терроризма является одним из вопросов борьбы с транснациональной организованной преступностью.

% Chưa biết có nên ghi hay không - Конвенция ООН против коррупции 2003 г. (статья 23)

% Конвенция СЕ об отмывании, выявлении, изъятии и конфискации доходов от преступной деятельности и о финансировании терроризма 2005 г. (9 статья)

\newpage

\section{Термины и определения}

\hfill

\subsection{Термины и определения правовой базы Вьетнама}

В \cite{law14} поясняются следующие термины и определения:

\begin{itemize}[label={--}]
    \item отмывание денег -- действия организаций или лиц по легализации происхождения имущества от преступления;
    \item доходы, полученные преступным путем -- имущество, полученное прямо или косвенно в результате совершения преступления; доходы, прибыль, прибыль, полученная от имущества, полученного преступными действиями;
    % \item операциями на крупные суммы, подлежащими отчетности, являются операции с наличными деньгами или наличной иностранной валютой, совершаемые один или несколько раз в день, общая стоимость которых равна или превышает установленный лимит.
    % \item инициатором является владелец счета или лицо, которое просит финансовое учреждение осуществить перевод электронных денег в случае не прохождения счета.
    % \item электронный перевод денег -- операция, совершаемая в электронном виде по запросу отправителя через финансовое учреждение с целью перевода определенной суммы денег бенефициару в финансовом учреждении бенефициара. Бенефициаром может быть инициатор.
    \item клиентами являются организации и физические лица, которые используют или намереваются использовать услуги и продукты, предоставляемые финансовыми учреждениями, организациями и физическими лицами, осуществляющими деятельность в соответствующих нефинансовых отраслях.
    % \item бенефициарный собственник - физическое лицо, имеющее фактическое право собственности на один или несколько активов и имеющее право контролировать клиентов, осуществляющих операции, связанные с активами, в пользу этого лица; -- физическое лицо, обладающее полномочиями управлять юридическим лицом или юридическим соглашением.
    % \item корреспондентские банковские отношения -- отношения, возникающие между банком одной страны или территории, предоставляющим банковские, платежные и другие услуг, и банком-партнером в другой стране и территории.
    % \item черный список -- список организаций и лиц, причастных к терроризму и финансированию терроризма, составленный Министерством общественной безопасности, а также список определенных организаций и лиц, причастных к распространению и финансированию распространения оружия массового уничтожения, возглавляемый и устанавливаемый Министерство национальной обороны в соответствии с положениями закона.
    % \item список предупреждений -- список организаций и частных лиц, созданный Государственным банком Вьетнама для предупреждения организаций и частных лиц с высоким риском отмывания денег.
    % \item финансовая целевая группа является межправительственной организацией, которая обнародует стандарты и способствует эффективному осуществлению правовых, нормативных и практических мер по борьбе с отмыванием денег, финансированием терроризма, финансированием распространения оружия массового уничтожения и другими связанными с этим опасностями, которые угрожают целостности мировая финансовая система.
    % \item юридическое соглашение -- соглашение в форме траста или иной аналогичной формы, установленное иностранным законодательством, позволяющее доверенной стороне получить передачу юридического права собственности на активы от поручающей стороны для эксплуатации, управления и надзора за активами. в пользу выгодоприобретателя или для целей, определенных в договоре.
    % \item банк-оболочка -- это банк, который не имеет физического присутствия в стране или на территории, на которой он учрежден и имеет лицензию, а также не связан и не контролируется каким-либо регулирующим органом. % Какой финансовый режим управляется и контролируется?
    % \item некоммерческой организацией является организация, действующая в некоммерческих целях, в том числе ассоциации, социальные фонды, благотворительные фонды, религиозные организации, иностранные неправительственные организации, которые созданы, зарегистрированы, действуют в соответствии с законодательством Вьетнама.
    % \item иностранными лицами, имеющими политическое влияние, являются лица, занимающие высокие должности в иностранных ведомствах, организациях и международных организациях.
\end{itemize}

\hfil

\subsection{Термины и определения правовой базы Российской Федерации}

В \cite{fz115} поясняются следующие термины и определения:

\begin{itemize}[label={--}]
  \item доходы, полученные преступным путем, - денежные средства или иное имущество, полученные в результате совершения преступления;
  \item легализация (отмывание) доходов, полученных преступным путем, - придание правомерного вида владению, пользованию или распоряжению денежными средствами или иным имуществом, полученными в результате совершения преступления;
  \item клиент - физическое или юридическое лицо, иностранная структура без образования юридического лица, находящиеся на обслуживании организации, осуществляющей операции с денежными средствами или иным имуществом.
\end{itemize}

% \hfill

% \subsection{Сравнение термин и определений на Вьетнамском и Русском языках}

% На основе терминологии и определений в сфере ПОД/ФТ показано, что есть некоторые общие моменты по смыслу, например, отмывание денег -- это легализация доходов, полученных преступным путем. Однако из-за особенностей языка и многих других факторов определения несколько различаются. К ним относятся, например, юридические и физические лица по законодательству РФ, а по законодательству Вьетнама -- учреждения, организации и физические лица.

\newpage

\section{Анализ банковского надзора и системы ПОД/ФТ во Вьетнаме}

\hfill

\subsection{История развития департамента по предотвращению отмывания денег}

В соответствии с \cite{ndcp7405}, Председатель Государственного банка подписал \cite{qdnh1002} о создании Информационного центра по предотвращению отмывания денег при Государственном банке Вьетнама. Соответственно, Информационный центр по предотвращению отмывания денег является структурным подразделением, имеет собственную печать для транзакций и выполняет функции в качестве координационного центра для получения и обработки информации и выполнения сопутствующих задач, указанных в \cite{ndcp7405}.

Чтобы удовлетворить требования по предотвращению отмывания денег в новой ситуации, Председатель Государственным банком издал Решение № 476/QĐ-NHNN от 7 марта 2007 года о создании Информационного центра по борьбе с отмыванием денег, соответственно, по предотвращению отмывания денег. Информационный центр является подразделением Государственного банка, имеет собственную печать и функционирует в качестве координационного центра для получения, обработки и предоставления информации о предотвращении и борьбе с отмыванием денег, а также помогает губернатору выполнять задачи, указанные в \cite{ndcp7405}.

Далее Информационный центр по предотвращению отмывания денег вместе с тремя другими подразделениями, а именно Банковской инспекцией, Департаментом банков и небанковских кредитных организаций и Департаментом кооперативных кредитных организаций, объединился в Агентство по банковской инспекции и надзору. Документ \cite{qdttg8309} Премьер-министра, регулирующее функции, задачи, полномочия и организационную структуру Агентства по банковской инспекции и надзору при Государственном банке Вьетнама. Информационного центр по предотвращению отмывания денег, было переименован в Департамент по предотвращению отмывания денег при Агентстве банковской инспекции и надзора.

В августе 2014 года Премьер-министр издал \cite{qdttg3514}, заменяющее \cite{qdttg8309}, регулирующее функции, задачи и полномочия и Агентство по банковской инспекции и  надзору при Государственном банке Вьетнама. Соответственно, Департамент по предотвращению отмывания денег оставался подразделением Агентства банковской инспекции и надзора.

В июне 2019 года Премьер-министр издал \cite{qdttg2019}, заменяющее \cite{qdttg3514}, регулирующее функции, задачи и полномочия Агентство по банковской инспекции и надзору при Государственном банке Вьетнама. Соответственно, Департамент по предотвращению отмывания денег оставался подразделением Агентства банковской инспекции и надзора.

\hfill

\subsection{Структура, функция, задачи и полномочия органа по банковской инспекции и надзору во Вьетнаме}

Согласно \cite{law46}, в статье 49 об органах банковского инспекции и надзора:

\begin{enumerate}
  \item Агентство банковской инспекции и надзора является подразделением в организационной структуре Государственного банка, выполняющим задачи банковской инспекции и надзора, предотвращения и борьбы с отмыванием денег.
  \item Премьер-министр должен конкретно определить организацию, задачи и полномочия Агентства банковской инспекции и надзора.
\end{enumerate}

В соответствии с \cite{ndcp2614} и \cite{ndcp4319}, банковская инспекция и надзор -- это орган государственной инспекции, организованный в систему, включающую в себя:

\begin{enumerate}
    \item Агентство по банковской инспекции и надзору Государственного банка.
    \item Инспекция и надзор за филиалами Государственного банка в губерниях и городах центрального подчинения (далее - инспекция и надзор за филиалами Государственного банка).
\end{enumerate}

Согласно \cite{qdttg2019}, определены функции, задачи, полномочия и организационная структура Агентства банковской инспекции и надзора при Государственном банке. 

Агентство по банковской инспекции и надзору является подразделением, эквивалентным Генеральному департаменту при Государственном банке, выполняющим функцию консультирования и оказания помощи Председателю Государственного банка в государственном управлении кредитными учреждениями, филиалами иностранных банков, государственном управлении по банковскому надзору, урегулированию жалоб и доносов, предупреждению и борьбе с коррупцией, предупреждению и борьбе с отмыванием денег, страхованию вкладов; проведению административных проверок, специализированных проверок и банковского надзора в сферах, находящихся в ведении Государственного банка; для предотвращения и борьбы с отмыванием денег и финансированием терроризма в соответствии с законом и по поручению Председателя Государственного банка.

Организационная структура Агентства банковского надзора и надзора состоит из 08 подразделений:

\begin{enumerate}
    \item Департамент административной инспекции, урегулирования жалоб и доносов и предупреждения и борьбы с коррупцией (далее -- Отдел I).
    \item Департамент политики в области банковской операционной безопасности (далее -- Отдел II).
    \item Офис.
    \item Департамент банковской инспекции и надзора I (далее -- Департамент I);
    \item Департамент банковской инспекции и надзора II (далее -- Департамент II).
    \item Департамент банковской инспекции и надзора III (далее -- Департамент III).
    \item Департамент надзора за безопасностью системы кредитных организаций (далее -- Департамент IV).
    \item Департамент по борьбе с отмыванием денег (далее -- Департамент V).
\end{enumerate}

Инспекция и надзор за филиалами Государственного банка является подразделением, входящим в организационную структуру филиалов Государственного банка, оказывающим помощь директору филиалов Государственного банка в управлении государством, проведении административных проверок, инспектировании и надзоре за банками, рассмотрении жалоб и доносов, предупреждение коррупции и борьба с ней, предотвращение и борьба с отмыванием денег, предотвращение и борьба с финансированием терроризма для субъектов банковского управления, инспекции и надзора в этой области в соответствии с назначением, децентрализацией и разрешением Председателя Государственным банком и в соответствии с положениями закона. В частности, Инспекция и надзор филиала Государственного банка Ханоя, Инспекция и надзор филиала Государственного банка города Хошимина являются подразделениями, эквивалентными подотделам филиала Государственного банка города Ханоя и филиала Государственного банка Хошимина.

Инспекция и надзор за деятельностью филиалов Государственного банка находятся под непосредственным руководством и руководством Директора филиалов Государственного банка, а также под руководством и руководством Агентства по банковской инспекции и надзору в отношении работы и осуществления банковского надзора и надзора, рассмотрения жалоб. выявление, предупреждение коррупции и борьба с ней, предупреждение и борьба с отмыванием денег, предупреждение и борьба с финансированием терроризма.

\hfill

\subsection{Структура, функция, задачи и полномочия банковского надзора в Российской Федерации}

В соответствии с \cite{fz86} (Статья 56):

\begin{enumerate}
  \item Банк России является органом банковского регулирования и банковского надзора. Банк России осуществляет постоянный надзор за соблюдением кредитными организациями, филиалами иностранных банков и банковскими группами законодательства Российской Федерации, нормативных актов Банка России, установленных ими обязательных нормативов и (или) установленных Банком России индивидуальных предельных значений обязательных нормативов. Банк России осуществляет анализ деятельности банковских холдингов и использует полученную информацию для целей банковского надзора за кредитными организациями, филиалами иностранных банков и банковскими группами, входящими в банковские холдинги.
  \item Главными целями банковского регулирования и банковского надзора являются поддержание стабильности банковской системы Российской Федерации и защита интересов вкладчиков и кредиторов. Банк России не вмешивается в оперативную деятельность кредитных организаций, филиалов иностранных банков, за исключением случаев, предусмотренных федеральными законами.
  \item Функции Банка России в сфере регулирования банковской деятельности и банковского надзора, установленные настоящим Федеральным законом, осуществляются через действующий на постоянной основе орган - Комитет банковского надзора, объединяющий руководителей структурных подразделений Банка России, обеспечивающих выполнение его надзорных функций.
\end{enumerate}

В сфере ПОД/ФТ/ФРОМУ ответственным органом является Росфинмониторинг под руководством Президента РФ \cite{pre808}. Его основное полномочие -- осуществляет контроль за соблюдением организациями, осуществляющими операции с денежными средствами или иным имуществом, и индивидуальными предпринимателями, указанными в \cite{fz115}, в сфере деятельности которых отсутствуют контрольные органы, законодательства РФ о ПОД/ФТ и исполнением решений, принимаемых по результатам мероприятий контроля, а также привлечение к ответственности указанных организаций, осуществляющих операции с денежными средствами или иным имуществом, и индивидуальных предпринимателей, допустивших нарушение законодательства РФ.

\hfill

\subsection{Сравнение банковского надзора и систем ПОД/ФТ во Вьетнаме и в Российской Федерации}

На основе анализа правовых и нормативных документов показывано, что банковский надзор осуществляется центральным банком каждой страны. В данном случае это Государственный банк Вьетнама и Центральный банк Российской Федерации.

Различия в правовых баз Вьетнама и Российской Федерации:

\begin{itemize}
  \item во Вьетнаме действуют два разных закона: один о предотвращении отмывания денег \cite{law14}, другой о борьбе с финансированием терроризма \cite{law28};
  \item в Российской Федерации есть федеральный закон о ПОД/ФТ \cite{fz115}.
\end{itemize}

В рамках международного сотрудничества Вьетнам и Российская Федерация приняли Конвенции ООН по предотвращению транснациональной преступности, включая отмывание денег и финансирование терроризма. Однако, в зависимости от географического положения Вьетнам не подписывает уставы таких организаций, как ЕС, СНГ, ... как Российская Федерация.

В сфере банковской инспекции и надзора при центральном банке каждой страны существует орган, выполняющий функции банковского надзора: Агентство банковской инспекции и надзора во Вьетнаме и Комитет банковской надзора в Российской Федерации.

В сфере ПОД/ФТ/ФРОМУ во Вьетнаме ответственным подразделением является Департамент по борьбе с отмыванием денег, находящийся в непосредственном подчинении Агентства по банковской инсекции и надзору \cite{qdttg2019}. Как упоминалось выше, Агентство по банковской инспекции и надзору принадлежит Государственному банку Вьетнама \cite{ndcp2614}. В этом отличие от Российской Федерации, где задача ПОД/ФТ/ФРОМУ возложена на Росфинмониторинг, действующий под руководством Президента РФ \cite{pre808}.

\newpage

\section*{Заключение}
\addcontentsline{toc}{section}{Заключение}

На основе проведения домашнего задания можно сделать следующие выводы:

\begin{enumerate}
  \item На основе подбора и анализа информационных источников показано, что доступ к информационным ресурсам не ограничен. Это позволяет искать, ссылаться и документировать, когда необходимо проанализировать и сравнить вопросы в сфере ПОД/ФТ.
  \item На основе подготовки перечня терминов и определений показано, что основые термины и определения в сфере ПОД/ФТ между Вьетнамом и Россией весьма схожи.
  \item На основе анализа особенностей банковского надзор и системы ПОД/ФТ во Вьетнама и в Российской Федерации показано, что существуют сходства в правовой базе и структуре системы ПОД/ФТ. Однако есть и некоторые различия в зависимости от государственного аппарата в каждой стране.
\end{enumerate}

\newpage

\addcontentsline{toc}{section}{Список используемых источников}
\printbibliography[env=gostbibliography, title={Список используемых источников}]

\end{document}